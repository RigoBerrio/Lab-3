\documentclass{article}
\usepackage[utf8]{inputenc}
\usepackage[spanish]{babel}
\usepackage{listings}
\usepackage{graphicx}
\graphicspath{ {images/} }
\usepackage{cite}

\begin{document}

\begin{titlepage}
    \begin{center}
        \vspace*{1cm}
        \Huge
        \textbf{Practica laboratorio N°3}
            
        \vspace{2 cm}
      
            
        \vspace{1.5cm}
            
        \textbf{Ivonne  Rosero }\\
        \large
        1007687589
        
        \vspace{2cm}
        \LARGE
        
        \textbf{Rigoberto Berrio}\\
        \large
        1040327583
            
        \vfill
             
        \vspace{0.8cm}
            
        \Large
        Despartamento de Ingeniería Electrónica y Telecomunicaciones\\
        Universidad de Antioquia\\
        Medellín\\
        Mayo de 2021
            
    \end{center}
\end{titlepage}
\tableofcontents
\newpage
\section{Requisitos para presentar la práctica}\label{intro}
\begin{itemize}
\item La práctica debe desarrollarse en parejas. \\
\end{itemize}

\begin{itemize}
\item Minimizar el uso de variables globales \\
\end{itemize}

\begin{itemize}
\item No se puede utilizar clases ni estructuras \\
\end{itemize}

\begin{itemize}
\item Debe resolverse el problema de codificación y el problema de decodificación  para cada método, ya que la aplicación del cajero automático exige tanto la  codificación como la decodificación.\\
\end{itemize}
 
\begin{itemize}
\item Deben utilizarse tanto arreglos de char como string, arreglos de char para el  primer método (codificación y decodificación) y string para el segundo  método (codificación y decodificación). \\
\end{itemize}

\begin{itemize}
\item El nombre del archivo sin codificar (y decodificado) será natural.txt, mientras  que el nombre del archivo codificado será codificado.dat, para ambos  métodos. \\
\end{itemize}

\begin{itemize}
\item El programa debe contar con un menú que permita seleccionar si se ejecuta  la codificación por el primer método, la decodificación por el primer método,  la codificación por el segundo método, la decodificación por el segundo método o la aplicación del cajero.\\ 
\end{itemize}

\begin{itemize}
\item El cajero debe contar con un menú propio que permita seleccionar las  distintas funcionalidades exigidas. 
\end{itemize}

\begin{itemize}
\item El cajero debe contar con 2 archivos de texto: sudo.txt, en el cual estará  guardada la clave encriptada del administrador y usuario.dat, en el cual  estarán guardados de forma encriptada los datos de los diferentes usuarios  del banco. No debe haber más archivos de texto. Estos datos pueden estar 
			codificados por cualquiera de los dos métodos. \\
\end{itemize}

\begin{itemize}
\item Ambos miembros del equipo deben entregar el archivo de la práctica, esto  con el fin de observar quienes faltaron por entregarla. \\
\end{itemize}

\begin{itemize}
\item En un comentario dentro de la tarea del classroom debe aparecer el nombre  del compañero con el que trabajó, para facilitar la revisión de la práctica.\\
\end{itemize}

\begin{itemize}
\item No utilizar librerías de manejo de contenedores (map, vector…) \\
\end{itemize}



\section{Método de Evaluación} \label{contenido}
La evaluación de esta primera práctica consta de 3 partes, de igual valor cada una: 
\begin{itemize}
\item Evaluación del código subido al classroom. Se revisará si está completo, si  implementa soluciones lógicas ordenadas, si es entendible, si utiliza  correctamente los diferentes tipos de datos y aprovecha las funciones y  librerías para la solución de sus problemas (Variables con nombres dicientes, 
		  comentarios, ...) 
\end{itemize}

\begin{itemize}
\item Se les mandará un texto a codificar y otro a decodificar el día de la sustentación, junto con el método y la semilla. Se evaluará la efectividad de los códigos en base a estos resultados. 
\end{itemize}

\begin{itemize}
\item Se les mandará un archivo sudo.txt y un archivo usuario.dat el día de la sustentación, junto con el método y la semilla. Se hará una sustentación oral  rápida del funcionamiento del cajero.
\end{itemize}






\section{Aprendizaje}\label{contenido}
\begin{itemize}
\item  Manipulación cadenas de caracteres en C++.
\end{itemize}
\begin{itemize}
\item La gestion de archivos en C++.
\end{itemize}
\begin{itemize}
\item Conocimiento de  cómo manejar Excepciones.
\end{itemize}


\end{document}
